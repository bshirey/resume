%% start of file `template.tex'.
%% Copyright 2006-2013 Xavier Danaux (xdanaux@gmail.com).
%
% This work may be distributed and/or modified under the
% conditions of the LaTeX Project Public License version 1.3c,
% available at http://www.latex-project.org/lppl/.


\documentclass[11pt,a4paper,sans]{moderncv}

% moderncv themes
\moderncvstyle{casual}
\moderncvcolor{green}

% adjust the page margins
%\photo[70pt][0.0pt]{photos/logo.png}
\usepackage[scale=0.75]{geometry}

% personal data
\name{Christopher Brian}{Shirey}
\title{Software Architect, Technical Leader, Innovator, Executive}
\address{600 Dream Catcher Dr.}{Leander, TX, 78641}{USA}
\phone[mobile]{+1~(321)~749~1780}
\email{Brian.Shirey@gmail.com}

\begin{document}

\makecvtitle
\section{Biography}
\cvitem{}{
Mr. Brian Shirey is a proven leader of both technical and non-technical teams, a solid executor, and innovator. He has successfully led teams of varying sizes, ranging from solo projects to 100-person development projects, and has held executive responsibilities for many of those projects. Mr. Shirey was the first employee of Security Innovations (SI) as a Software Engineer. Mr. Shirey helped grow a small team that was acquired by Raytheon in 2008, served as the Technical Director/CTO of that company, and ultimately as the Director/CEO of SI Government Solutions (SIGovs) and two other Raytheon acquisitions, collectively known as Raytheon Centers of Innovation (COI). As Director of COI, Mr. Shirey led 320+ employees in 10 locations in 9 US States, including 275 computer security engineers and covering over \$85M per year in revenue for US Government agencies.}
\cvitem{}{
Mr. Shirey currently acts as both the global Chief Architect as well as the Vice President of Innovation Labs at Forcepoint. In these roles, Mr. Shirey is responsible for generating research projects focused on solving customer needs, generating new intellectual property, performing research and prototypes, feeding those ideas and prototypes into the product roadmaps, and ensuring a modern, consistent platform architecture for all Forcepoint products while interacting with all product-focused functional areas to ensure business alignment with the research projects.}
\cvitem{}{
Prior to his current role, Mr. Shirey has acted in several additional roles within Forcepoint. Mr. Shirey joined Forcepoint as the Senior Vice President of Research and Development. In this role, Mr. Shirey was responsible for leading the Forcepoint Security Labs - a team of nearly 70 security researchers focused on securing Forcepoint's customers through intelligent and automated security investigations of customer web and email traffic, totalling billions of customer messages per year. Mr. Shirey concurrently led the Advanced Research Team, a team dedicated to perform state-of-the-art research and develop new technologies in the security space that were aimed at becoming future products. Within Forcepoint, Mr. Shirey has also served as the acting Vice President of Research and Development for the Data and Insider Threat Security business unit.}
\cvitem{}{
Mr. Shirey has been the lead engineer for a number of software tools and projects. These projects include a number of CNO Tools for the US Government, Vulnerability Discovery tools (fuzzers, test deployment/management, and vulnerability scanners), low-level libraries for biomedical equipment, and web-based applications.}
\cvitem{}{
In addition to his technical experience, Mr. Shirey has a strong academic background, including a Masters degree in computer science from Florida Institute of Technology (2004) that focused on the epidemiology of malicious code spread on computer networks. Prior to and during his graduate program he developed a Windows vulnerability scanner product as well as a behavior-based intrusion prevention product.}

\section{Professional Experience}
\cventry{2017-present}{Vice President of Innovation Labs}{Forcepoint}{Austin, TX, USA}{}{
Mr. Shirey leads the company-wide innovation vehicle, partnering with all the BUs, the Product Management, Strategy, and Corporate teams to research and prototype future capabilities, technologies, and services with the goal of driving future growth. Mr. Shirey also encourages and sponsors research throughout the company, driving a culture of innovation.}
\cventry{2016-present}{Chief Architect}{Forcepoint}{Austin, TX, USA}{}{
Mr. Shirey is responsible for high-level architecture globally across the various business units within Forcepoint, working with architects and CTOs of each business unit to ensure alignment across the company, modern and standard technologies are employed, and code reuse is optimized whereever possible.}
\cventry{2015-2016}{Senior Vice President of Research and Development}{Forcepoint}{Melbourne, FL, USA}{}{
Mr. Shirey was responsible for Security Labs at Forcepoint - focusing on security research, custom reverse engineering and forensics of malware, development of common security components used across all Forcepoint products, and security efficacy for the company. He also founded and led the Advanced Research Team, which focused on cutting edge security research that could become products in 18-24 months - creating prototypes and working with product teams to turn them into sellable products.}
\cventry{2002--2015}{Multiple Positions}{Raytheon SI Government Solutions}{Indialantic, FL, USA}{}{
During his 12+ years at SIGovs, Mr. Shirey served in a number of leadership roles of increasing responsibility, including: commercial product development lead, project technical lead, CNO Tool development lead, Engineering Lead, Chief Technical Officer (CTO), and CEO/Director. During this period, SIGovs routinely grew 15\% or more in revenue each year, and was recognized as industry leaders in vulnerability discovery and CNO tool development.\newline\newline
As Director of COI, Mr. Shirey led 320+ employees in 10 locations in 9 US States, including 275 computer security engineers and covering over \$75M per year in revenue for US Government agencies. In his previous role of CTO for Raytheon COI, Mr. Shirey was ultimately responsible for all engineering technical activities, setting technical direction for the organization, and determining how and where to spend IRAD funds. Mr. Shirey was influential in setting direction and organizational goals related to cyber security, collaborating with business area leaders across Raytheon. He directly oversaw \$8M of IRAD investment, and transitioned discriminators into customer programs and products throughout Raytheon. In this role, Mr. Shirey worked closely with business development, program management, finance, contracts, human resources, and senior leadership to ensure mission success. Mr. Shirey also started the SI Emergent Ventures Engine (SIEVE) team to rapidly identify and develop game-changing technologies in relevant OCO business areas.\newline\newline}
\cventry{2002--2015}{Multiple Positions (continued)}{Raytheon SI Government Solutions}{Indialantic, FL, USA}{}{
Prior to this work, Mr. Shirey served as the project lead or senior developer for several customer-funded development projects and a number of vulnerability research tools.  In these efforts, Mr. Shirey worked closely with the Prime and Sponsor through the entire software development lifecycle -- developing CONOPS, requirements clarification, initial design, implementation, and testing strategy.\newline\newline
As the Commercial Tools Development Manager for Security Innovation, Mr. Shirey designed and developed the Holodeck runtime fault injection technology product. He also defined, implemented, and refined software process and methodology throughout the organization.\newline
}
\cventry{1997--2002}{Software Engineer}{Harris Corporation}{Melbourne, FL, USA}{}{
At Harris Corporation, Mr. Shirey was a developer on both the STAT Scanner and STAT Neutralizer projects at Harris Corporation. He also designed, developed, and administered a Solaris-based internal website used to reduce reliance on paper-based processes for printed circuit board design.\newline
}
\cventry{1999--2000}{Software Engineer}{Ciber Inc}{Melbourne, FL, USA}{}{
As a Software Engineer at Ciber, Mr. Shirey developed APIs for transmitting data over the parallel and Ethernet ports in C for use in biomedical devices (heart monitoring machines) running MS-DOS. He also developed 16-bit SVGA graphics library in C for MS-DOS.\newline
}

\section{Areas of Expertise}
\cvitem{\textbf{Languages}}{C, C++, Python, Java, C\#, x86 assembly, HTML, Javascript}
\cvitem{\textbf{APIs}}{Win32, Windows Kernel APIs, .NET, Java}
\cvitem{\textbf{Concepts}}{OOA/OOD (UML, Design Patterns), Networking, Secure Programming,
Reverse Engineering, Malware Analysis, CNO Tools
}
\cvitem{\textbf{Platforms}}{Windows (2000, NT, XP, Vista, 7), Linux (Ubuntu), Mac OSX}
\cvitem{\textbf{Tools}}{Visual Studio, GCC, Vim, SVN/GIT, \LaTeX }

\section{Clearances}
\cventry{2005--Present}{Top Secret, Full-Scope Polygraph}{}{}{}{}

\section{Education}
\cventry{1997--2000}{Bachelor of Science, Computer Science}{Florida Institute of Technology}{Melbourne, FL, USA}{}{}
\cventry{2000--2004}{Master of Science, Computer Science}{Florida Institute of Technology}{Melbourne, FL, USA}{}{}
%\cventry{year--year}{Degree}{Institution}{City}{\textit{Grade}}{Description}

\section{Master thesis}
\cvitem{title}{\emph{Modeling the Spread and Prevention of Malicious Mobile Code Via Simulation}}
\cvitem{supervisors}{Dr. Richard Ford}

\clearpage
\end{document}